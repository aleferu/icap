\section{Anexo 1: Replicación}

Todo se despliega a través de CloudFormation, haciendo uso de su sistema de parámetros y salidas. Además, todas las pilas se llamarán igual que el fichero de la plantilla, sin la extensión \texttt{.yaml} y con el prefijo \texttt{proy-}. Para cada plantilla hay que configurar el rol de IAM como \textit{LabRole}.

\subsection{proy-databases}

El primer fichero que se sube se llama \texttt{databases.yaml}. Este fichero se encarga de crear el bucket que recibe los ficheros, la función Lambda que procesa dichos ficheros, la tabla de \textit{DynamoDB} y el tópico de \textit{SNS}. Los dos parámetros que necesita son:

\begin{itemize}
     \item \underline{IAMRoleName}: Nombre del rol que se asociará a la función Lambda.
     \item \underline{EmailTarget}: Correo al que se enviará una alerta cuando se encuentre una desviación mayor a $0.5$.
\end{itemize}

Se recomienda dejar \textit{IAMRoleName} con su valor por defecto, mientras que el email se puede configurar como se guste (mientras sea válido).

No hace falta esperar a que esta plantilla termine para continuar, pues esta plantilla es completamente independiente al resto.

\subsection{proy-network}

El segundo fichero es \texttt{network.yaml}. Esta plantilla se encarga de crear la infraestructura de red que dará soporte a las instancias EC2. En concreto:

\begin{itemize}
    \item VPC.
    \item Internet Gateway.
    \item Tabla de enrutamiento con acceso a internet.
    \item Tabla de enrutamiento para las subredes privadas.
    \item Tres subredes públicas, cada una en una zona distinta.
    \item Tres subredes privadas, cada una en una zona distinta.
    \item Un GW NAT.
    \item Dos grupos de seguridad, uno para el balanceador de carga y otro para las instancias EC2 y la tarea.
    \item El repositorio ECR que contendrá el contenedor Docker (creado en el siguiente paso).
    \item Todo lo necesario para que todo lo anterior se despliegue adecuadamente.
\end{itemize}

Esta plantilla no tiene parámetros, pero sí que tiene salidas. Las salidas son los IDs de la VPC, las subredes, y de los grupos de seguridad.

Los siguientes pasos necesitan del funcionamiento de esta pila.

\subsection{proy-testinstance}

El tercer fichero es \texttt{testinstance.yaml}. Esta plantilla se encarga de crear una instancia que construirá el contenedor Docker y la subirá al contenedor creado en la plantilla anterior. La plantilla necesita tres parámetros:

\begin{itemize}
    \item \underline{ECRPassword}: Contraseña utilizada para el acceso a ECR. Normalmente sigue la estructura siguiente:
    \begin{itemize}
        \item \texttt{ACCOUNTID.dkr.ecr.REGION.amazonaws.com} (lo que habría que cambiar son los doce dígitos del principio).
    \end{itemize}
    \item \underline{ECRUsername}: Igual para todos, así que no habría que cambiarlo. Se ha dejado como parámetro por si esto lo cambian en el futuro.
    \item \underline{IAMProfile}: Perfil IAM para la instancia. El valor por defecto es válido.
\end{itemize}

Una vez que sea seguro que el contenedor está subido, esta pila se puede eliminar, pues ya habría cumplido su propósito y no se necesita su existencia de aquí en adelante. Eso sí, no continuar con el último paso hasta que el contenedor se haya subido, pues es necesario.

\subsection{proy-cluster}

El cuarto y último fichero es \texttt{cluster.yaml}. Esta plantilla se encarga de crear la tarea, los servicios, el balanceador de carga y el cluster ECS. Los parámetros se deben dejar por defecto (a no ser de que se sepa lo que se está haciendo).

\begin{itemize}
    \item \underline{LatestECSOptimizedAMI}: ID de una AMI válida.
    \item \underline{IAMProfile}: Perfil IAM usado para la plantilla de despliegue del clúster.
    \item \underline{IAMRoleName}: Nombre del rol IAM usado en la definición de la tarea.
\end{itemize}

Tarda un poco en desplegarse, pero cuando termine ya se habrá terminado de crear todo lo necesario para le entrega.
